% Decouverte de l'IDE QTcreator
%
% Auteur  : Gregory DAVID
%%

\documentclass[12pt,a4paper,oneside,titlepage,final]{article}

\usepackage{groolot_layout}
\usepackage{groolot_glossaire}
\usepackage{groolot_commands}
\usepackage{flafter}

\newcommand{\MONTITRE}{Programmation Événementielle}
\newcommand{\MONSOUSTITRE}{Découverte de l'\glsentrytext{IDE} \gls{QtCreator}}
\newcommand{\DISCIPLINE}{\glsentrytext{SiSIX} -- \glsentrydesc{SiSIX}}

\usepackage[%
	pdftex,%
	pdfpagelabels=true,%
	pdftitle={\MONTITRE},%
	pdfauthor={Grégory DAVID},%
	pdfsubject={\MONSOUSTITRE},%
        colorlinks,%
]{hyperref}

\title{
	\begin{tabular*}{\linewidth}{@{\extracolsep{\fill}}lr}
		& {\Huge {\bf \MONTITRE}} \\
		& {\huge \MONSOUSTITRE} \\
		& {\large \DISCIPLINE} \\
	\end{tabular*}
}

%\printanswers\groolotPhiligranne{CORRECTION}

\begin{document}
% Page de titre
\maketitle

% Copyright
\include{copyright}

% Contenu
\HEADER{\MONTITRE}{\DISCIPLINE}

L'ensemble de la documentation de \gls{QtCreator} peut être consultée
en ligne : voir \cite{QTCREATOR_MANUAL}. Autrement, voici un rapide
descriptif de l'application.

\section{Interface graphique de l'\glsentrytext{IDE} \gls{QtCreator}}
La \gls{GUI} de \gls{QtCreator} (voir \figurename
\vref{fig:qtcreatorGUI}) est scindée en plusieurs zones, indépendantes
du mode dans lequel nous nous trouvons :
\begin{itemize}
    \item Le sélecteur de mode : \figurename
    \vref{fig:qtcreatorGUImodeSelector}
    \item Le sélecteur de production : \figurename
    \vref{fig:qtcreatorGUIkitSelector}
    \item La zone de compilation, de débogage et d'exécution :
    \figurename \vref{fig:qtcreatorGUIrunDebugCompile}
    \item Le moteur de recherche : \figurename
    \vref{fig:qtcreatorGUIsearchEngine}
    \item Les panneaux de sorties (informations variées) : \figurename
    \vref{fig:qtcreatorGUIoutputPanes}
    \item Construire un nouveau projet (enfin) : \figurename
    \vref{fig:qtcreatorGUInewProject}
\end{itemize}

\image{data/qt-creator_GUI.png}{width=15cm}{\gls{GUI} de l'\gls{IDE}
  \gls{QtCreator}}{fig:qtcreatorGUI}

\image{data/qt-creator_GUI_modeSelector.png}{width=10cm}{Le sélecteur
  de mode}{fig:qtcreatorGUImodeSelector}

\image{data/qt-creator_GUI_kitSelector.png}{width=10cm}{Le sélecteur
  de production}{fig:qtcreatorGUIkitSelector}

\image{data/qt-creator_GUI_run-debug-compile.png}{width=10cm}{Les
  commandes d'exécution, de débogage et de
  compilation}{fig:qtcreatorGUIrunDebugCompile}

\image{data/qt-creator_GUI_searchEngine.png}{width=10cm}{Le moteur de
  recherche}{fig:qtcreatorGUIsearchEngine}

\image{data/qt-creator_GUI_outputPanes.png}{width=10cm}{Les panneaux
  de sorties}{fig:qtcreatorGUIoutputPanes}

\image{data/qt-creator_GUI_newProject.png}{width=10cm}{Créer un
  nouveau projet}{fig:qtcreatorGUInewProject}

\subsection{Création d'un nouveau projet}
\label{sec:CreationNewProject}

\subsubsection{Choix du modèle pour l'application à construire}
La boîte de dialogue \figurename
\vref{fig:qtcreatorGUInewProjectModel} permet de choisir le modèle de
l'application à construire. Le modèle est un moyen de configurer
spécifiquement l'\gls{IDE} et les composants afin d'optimiser la
construction et la compilation du projet.

\image{data/qt-creator_GUI_newProject_model.png}{width=12cm}{Définir
  le modèle d'application à
  construire}{fig:qtcreatorGUInewProjectModel}

\subsubsection{Choix du lieu de stockage du projet}
La boîte de dialogue \figurename
\vref{fig:qtcreatorGUInewProjectLocation} permet de spécifier le lieu
de stockage du projet.

Assurez-vous d'avoir les droits suffisants sur le répertoire parent.

\image{data/qt-creator_GUI_newProject_location.png}{width=12cm}{Définir
  le lieu de stockage du projet}{fig:qtcreatorGUInewProjectLocation}

\subsubsection{Détails du projet}
La boîte de dialogue \figurename
\vref{fig:qtcreatorGUInewProjectDetails} vous donne l'occasion de
paramétrer le nom de la classe principale permettant de construire une
fenêtre graphique. Mettez encore une fois un nom de classe qui soit
cohérent avec votre projet.

\image{data/qt-creator_GUI_newProject_details.png}{width=12cm}{Donner
  les éléments détaillés du projet}{fig:qtcreatorGUInewProjectDetails}

\subsubsection{Choix du kit de production}
La boîte de dialogue \figurename \vref{fig:qtcreatorGUInewProjectKit}
vous permet de choisir le mode de production de votre appplication
(dans un environnement paramétré par défaut, il n'y a pas d'autre
choix que le choix par défaut \emph{Desktop}).

\image{data/qt-creator_GUI_newProject_kit.png}{width=12cm}{Sélectionner
  le kit de production du projet}{fig:qtcreatorGUInewProjectKit}

\subsubsection{Choix du gestionnaire de version \gls{Git}}
La boîte de dialogue \figurename \vref{fig:qtcreatorGUInewProjectGit}
vous permet de sélectionner \gls{Git} en tant que gestionnaire de
version.

Il va de soit qu'il n'est pas envisageable que vous n'utilisiez pas de
dépôt \gls{Git} pour chacun de vos projets.

\image{data/qt-creator_GUI_newProject_versioningGit.png}{width=12cm}{Faire
  suivre le projet avec \gls{Git}}{fig:qtcreatorGUInewProjectGit}

\subsubsection{Projet créé, en avant !}
La \figurename \vref{fig:qtcreatorGUInewProjectDone} prséente
l'interface d'\emph{édition}, rendue accessible à la fin de la
création de votre projet.

Vous allez enfin pouvoir \textbf{programmer}.

\image{data/qt-creator_GUI_newProject_done.png}{width=15cm}{Interface
  d'édition du code de l'application une fois la création
  achevée}{fig:qtcreatorGUInewProjectDone}

\subsection{Concevoir une \gls{GUI}}
\label{sec:concevoirGUI}
Dès lors que le projet est créé (voir \vref{sec:CreationNewProject}),
nous pouvons voir qu'il existe un formulaire par défaut de
créé. L'ouverture, par double clic, dudit formulaire ouvre le mode
\emph{Design} dans l'\gls{IDE}, tel que présenté dans la \figurename
\vref{fig:qtcreatorGUIDesign}.

\image{data/qt-creator_GUI_Design.png}{width=15cm}{Mode de création
  d'interface graphique}{fig:qtcreatorGUIDesign}

\subsubsection{Espace de création du contenu du formulaire}
\label{sec:concevoirGUIformulaire}
Voir \figurename \vref{fig:qtcreatorGUIDesignFormulaire}.

\image{data/qt-creator_GUI_Design_windowDesign.png}{width=15cm}{Espace
  de création du contenu du
  formulaire}{fig:qtcreatorGUIDesignFormulaire}

\subsubsection{Catalogue des \emph{widgets} intégrables dans le
  formulaire}
\label{sec:concevoirGUIwidgetsCatalogue}
Voir \figurename \vref{fig:qtcreatorGUIDesignCatalogue}.

\image{data/qt-creator_GUI_Design_widgetBox.png}{width=15cm}{Catalogue
  des \emph{widgets disponibles}}{fig:qtcreatorGUIDesignCatalogue}

\subsubsection{Construction du formulaire par glisser-déposer}
\label{sec:concevoirGUIDragDrop}
Voir \figurename \vref{fig:qtcreatorGUIDesignDragDrop}.

\image{data/qt-creator_GUI_Design_DragDropWidget.png}{width=15cm}{La
  construction du formulaire se fait par glisser-déposer des
  \emph{widgets} issus du catalogue}{fig:qtcreatorGUIDesignDragDrop}

\subsubsection{Liste des widgets insérés dans le formulaire}
\label{sec:concevoirGUIwidgetsList}
Chacun des \emph{widget} inséré dans le formulaire est listé dans la
zone identifiée dans la \figurename
\vref{fig:qtcreatorGUIDesignWidgetsList}.

\image{data/qt-creator_GUI_Design_widgetsList.png}{width=15cm}{Liste
  des \emph{widgets} insérés dans le formulaire
  courant}{fig:qtcreatorGUIDesignWidgetsList}

\subsubsection{Propriétés des widgets}
\label{sec:concevoirGUIwidgetProperties}
Chaque objet \gls{Qt} possède des propriétés propres et d'autres qui
lui sont héritées de son(ses) parent(s). Chacune de ses propriétés est
modifiable à la main depuis la zone présentée en \figurename
\vref{fig:qtcreatorGUIDesignWidgetProperties} ou programmatiquement
(seulement si l'attribut est public, sinon il faut passer par un
accesseur).

\image{data/qt-creator_GUI_Design_widgetProperties.png}{width=15cm}{Zone
  des propriétés du \emph{widget}
  courant}{fig:qtcreatorGUIDesignWidgetProperties}


\section{Travail À Faire}
L'objectif est de concevoir une application ressemblant à la
\figurename \vref{fig:appObjectif} et ayant le comportement tel
qu'identifiable dans la vidéo accessible à l'adresse suivante :
\url{http://youtu.be/YiZIAqVtTRc}.
\begin{enumerate}
    \item Concevoir un projet suivant le modèle \emph{Application
      \gls{Qt} avec widgets} ;
    \item Ajouter les \emph{widgets} nécessaires permettant de
    reproduire l'application proposée ;
    \item Mettre en œuvre les comportements visibles sur la vidéo de
    présentation.
\end{enumerate}

\image{data/application_a_reproduire.png}{width=10cm}{Design de
  l'application à reproduire}{fig:appObjectif}

% References bibliographiques
\newpage \printbibheading
\printbibliography[nottype=online,check=notonline,heading=subbibliography,title={Bibliographiques}]
\printbibliography[check=online,heading=subbibliography,title={Webographiques}]

\printglossaries

\end{document}


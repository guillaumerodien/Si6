% Organisation des cours Si6
%
% Auteur  : Gregory DAVID
%%

\documentclass[12pt,a4paper,oneside,titlepage,final]{article}

\usepackage{groolot_layout}
\usepackage{groolot_glossaire}

\newcommand{\MONTITRE}{Organisation des cours \glsentrytext{SiSIX}}
\newcommand{\MONSOUSTITRE}{}
\newcommand{\DISCIPLINE}{\glsentrytext{SiSIX} -- \glsentrydesc{SiSIX}}

\usepackage[%
	pdftex,%
	pdfpagelabels=true,%
	pdftitle={\MONTITRE},%
	pdfauthor={Gr\'egory DAVID},%
	pdfsubject={\MONSOUSTITRE},%
        colorlinks,%
]{hyperref}
\usepackage{groolot_commands}

\title{
	\begin{tabular*}{\linewidth}{@{\extracolsep{\fill}}lr}
		& {\Huge {\bf \MONTITRE}} \\
		& {\huge \MONSOUSTITRE} \\
		& {\large \DISCIPLINE} \\
	\end{tabular*}
}

%\printanswers\groolotPhiligranne{CORRECTION}

\begin{document}
% Page de titre
\maketitle

% Copyright
\include{copyright}

% Contenu
\HEADER{\MONTITRE}{\DISCIPLINE}

\section{Référentiel}
\subsection*{Activités support de l'acquisition des compétences}

\paragraph{D1.1 -- Analyse de la demande}
\begin{itemize}
    \item A1.1.1 Analyse du cahier des charges d'un service à produire
\end{itemize}

\paragraph{D1.2 -- Choix d'une solution}
\begin{itemize}
    \item A1.2.4 Détermination des tests nécessaires à la validation
    d'un service
\end{itemize}

\paragraph{D1.3 -- Mise en production d'un service}
\begin{itemize}
    \item A1.3.1 Tests d'intégration et d'acceptation d'un service
    \item A1.3.3 Accompagnement de la mise en place d'un nouveau
    service
\end{itemize}

\paragraph{D4.1 -- Conception et réalisation d'une solution
  applicative}
\begin{itemize}
    \item A4.1.2 Conception ou adaptation de l'interface utilisateur
    d'une solution applicative
    \item A4.1.3 Conception ou adaptation d'une base de données
    \item A4.1.7 Développement, utilisation ou adaptation de
    composants logiciels
    \item A4.1.8 Réalisation des tests nécessaires à la validation
    d'éléments adaptés ou développés
    \item A4.1.9 Rédaction d'une documentation technique
    \item A4.1.10 Rédaction d'une documentation d'utilisation
\end{itemize}

\paragraph{D5.2 -- Gestion des compétences}
\begin{itemize}
    \item A5.2.1 Exploitation des référentiels, normes et standards
    adoptés par le prestataire informatique
    \item A5.2.2 Veille technologique
\end{itemize}

\subsection*{Savoirs-faire}
\begin{itemize}
    \item Concevoir une interface utilisateur
    \item Interpréter un schéma de base de données
    \item Développer et maintenir une application exploitant une base
    de données partagée\footnote{hormis les techniques de gestion
      concurrente des accès}
    \item Élaborer un jeu d'essai
    \item Valider et documenter une application
    \item Rédiger une documentation d'utilisation
    \item Utiliser des outils de travail collaboratif
\end{itemize}

\subsection*{Savoirs associés}
\begin{itemize}
    \item Architectures applicatives : concepts de base
    \item Techniques de présentation des données et des documents
    \item Interfaces homme-machine
    \item Fonctionnalités d'un outil de développement rapide
    d'applications
    \item Typologie des tests
    \item Techniques de mise au point
    \item Bonnes pratiques de documentation d'une application
    \item Techniques de rédaction d'une documentation d'utilisation
\end{itemize}


\section{Sélections et orientations}
Bien que cela ne soit ni une préconisation nationale, ni académique,
chaque savoir-faire se verra être potentiellement qualifié
différemment en fonction de la spécialité choisie par l'étudiant :
\gls{SLAM} ou \gls{SISR}.

\subsection{Concevoir une interface utilisateur}
\subsubsection{Pour les \gls{SLAM}}
QTCreator \& C++
\subsubsection{Pour les \gls{SISR}}
HTML5 \& PHP

\subsection{Interpréter un schéma de base de données}

\subsection{Développer et maintenir une application exploitant une
  base de données partagée}
\subsubsection{Pour les \gls{SLAM}}
\subsubsection{Pour les \gls{SISR}}

\subsection{Élaborer un jeu d'essai}
\subsubsection{Pour les \gls{SLAM}}
\subsubsection{Pour les \gls{SISR}}

\subsection{Valider et documenter une application}
MarkDown sur GitHub \& Doxygen \& ReadTheDocs

\subsection{Rédiger une documentation d'utilisation}

\subsection{Utiliser des outils de travail collaboratif}



% References bibliographiques
\newpage \printbibheading
\printbibliography[nottype=online,check=notonline,heading=subbibliography,title={Bibliographiques}]
\printbibliography[check=online,heading=subbibliography,title={Webographiques}]

\printglossaries

\end{document}

